 % problématique spécifique du PTA
  % parler de qu'est que va contituer le travail et quest que c'est la contribution genéré 
  % falar que ao longo do trabalho vamos fazer primeiro uma simulação de uma máquina assíncrona tradicional (simulaçção e controle) e que então fazendo a tensão no rotor ser diferente de zero podemos simular uma máquina do tipo MADA

  L'intégration efficace des énergies renouvelables, telles que l'éolien et l'hydraulique, dans le réseau électrique est essentielle pour la transition énergétique mondiale. La Chine et l'union européenne illustrent cet engagement par des initiatives ambitieuses visant à augmenter significativement leur capacité renouvelable, avec un objectif de 1200 GW pour la Chine d'ici 2030 \cite{ChinaWindIntegration} et le plan "Fit for 55" de l'UE pour réduire les émissions de CO2 de 55 \% d'ici 2030 \cite{EuropeEnergyTransition}. Ces efforts soulignent l'importance croissante de l'énergie verte dans le mix énergétique global.

Au cœur de cette transition, la Machine Asynchrone à Double Alimentation (MADA) se distingue par sa capacité à ajuster la puissance réactive, cruciale pour la stabilité du réseau électrique. Elle permet une meilleure intégration des sources renouvelables fluctuantes grâce à ses capacités de régulation, en offrant une inertie virtuelle et un amortissement réactif \cite{Qi2023DFIG}.

Parallèlement, la machine asynchrone à cage, pour sa part, est valorisée pour sa robustesse, sa simplicité et son faible besoin en maintenance, caractéristiques découlant de son rotor en cage d'écureuil. Cette machine trouve sa place dans de multiples applications industrielles, démontrant sa flexibilité et sa fiabilité \cite{Qi2023DFIG}.

Enfin, l'adoption de la modélisation et de la simulation pour étudier les machines électriques joue un rôle prépondérant dans l'optimisation de l'intégration des énergies renouvelables. Ces outils offrent une compréhension détaillée des dynamiques des machines et permettent le développement de stratégies de contrôle avancées, facilitant ainsi une transition énergétique efficace et innovante.

    \section{Objectives}

   Ce travail vise à simuler des machines asynchrones, à cage et à double alimentation, via MATLAB/Simulink \cite{MATLAB} \cite{Simulink}, en exploitant leurs équations dynamiques dans l'environnement Simulink. Pour la machine à cage, des régulateurs de courant et de vitesse seront développés pour ajuster sa vitesse, tandis qu'un contrôleur de puissance sera implémenté pour la machine à double alimentation afin de gérer son facteur de puissance dans un réseau simulé.