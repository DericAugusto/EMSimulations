
Cette étude s'est concentrée sur la simulation et l'analyse des machines asynchrones à cage et à double alimentation (MADA), explorant les méthodes de contrôle et leur application dans des systèmes de conversion d'énergie. L'usage de MATLAB/Simulink a permis d'examiner les performances des techniques de contrôle vectoriel pour réguler efficacement la vitesse des machines asynchrones à cage et la gestion de la puissance pour les MADA.

L'application des transformations de coordonnées, telles que Park, Clarke, et Concordia, a facilité la modélisation des dynamiques électriques et mécaniques des machines, permettant une analyse plus claire des stratégies de contrôle. Ces outils de modélisation ont été essentiels pour ajuster les paramètres de contrôle et pour identifier des voies d'amélioration des performances des machines étudiées.

Les simulations ont mis en évidence l'utilité des environnements logiciels pour tester divers paramètres et configurations, offrant un moyen économique et flexible d'évaluer les performances sans les contraintes des tests physiques. Cette approche a été bénéfique pour l'optimisation des systèmes de contrôle et la validation des concepts théoriques.

Néanmoins, les simulations ont également révélé certaines limitations, comme les difficultés liées à la résolution de boucles algébriques et les défis de convergence des solveurs. Ces problématiques ont souligné l'importance d'une modélisation et d'une sélection de paramètres attentives pour garantir des résultats fiables et représentatifs.

En somme, l'étude a apporté des contributions significatives à la compréhension des machines asynchrones à cage et à double alimentation, tout en reconnaissant la nécessité de poursuivre la recherche pour affiner les techniques de contrôle et exploiter pleinement les avantages des simulations numériques dans le développement de systèmes énergétiques avancés.